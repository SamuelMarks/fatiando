\section{Problemas lineares}

Se a função $f_i(\vect{p})$ (equação \ref{eq:fi}), que relaciona o vetor de
parâmetros $\vect{p}$ ao $i$-ésimo dado predito, for uma combinação linear dos
parâmetros, ela terá a seguinte forma

\begin{equation}
f_i(\vect{p}) = g_{i1}p_1 + g_{i2}p_2 + \dotsb + g_{iM}p_M + b_i \thinspace ,
\label{eq:comb_linear}
\end{equation}

\noindent em que $g_{ij}$, $j = 1, 2, \cdots, M$, e $b_i$ são constantes.
Portanto, a derivada de $f_i(\vect{p})$ em relação ao $j$-ésimo parâmetro
$p_j$ é dada por

\begin{equation}
\dfrac{\partial f_i(\vect{p})}{\partial p_j} = g_{ij} \thinspace ,
\end{equation}

\noindent que não depende dos parâmetros. Para um conjunto de $N$ dados, o vetor
de dados preditos tem a seguinte forma

\begin{equation}
\vect{f}(\vect{p}) = \mat{G}\vect{p} + \vect{b} \thinspace ,
\end{equation}

\noindent sendo $\mat{G}$ a matriz de sensibilidade (equação \ref{eq:jacobian})
e $\vect{b}$ um vetor de constantes.
Substituindo a expressão acima no gradiente da função do ajuste (equação
\ref{eq:gradphi}), chegamos a

\begin{equation}
\vect{\nabla}\phi(\vect{p}) = -2\mat{G}^T \left(\vect{d}^{\thinspace o} -
    \mat{G}\vect{p} - \vect{b} \right).
\end{equation}

\indent Seja $\opt{p}$ o vetor que minimiza a função $\phi(\vect{p})$, o
gradiente desta função avaliado em $\opt{p}$ é igual ao vetor nulo.
Isto nos leva ao sistema de equações

\begin{equation}
\mat{G}^T\mat{G}\opt{p} = \mat{G}^T\left(\vect{d}^{\thinspace o} - \vect{b} \right).
\label{eq:sistema_normal}
\end{equation}

\noindent Este sistema é conhecido como {\it sistema de equações normais}. A
solução para este sistema é

\begin{equation}
\opt{p} = (\mat{G}^T\mat{G})^{-1} \mat{G}^T
    \left(\vect{d}^{\thinspace o} - \vect{b} \right),
\label{eq:estimador_mq}
\end{equation}

\noindent que é conhecido como {\it estimador de mínimos quadrados}.
\\
\indent As equações \ref{eq:sistema_normal} e \ref{eq:estimador_mq} mostram que 
o vetor $\opt{p}$ que minimiza a função $\phi(\vect{p})$ pode ser
obtido diretamente a partir da matriz $\mat{G}$ e dos vetores $\vect{b}$ e
$\vect{d}^{\thinspace o}$. Isto caracteriza um {\it problema inverso linear}.
