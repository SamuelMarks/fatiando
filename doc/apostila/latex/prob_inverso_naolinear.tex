\section{Problemas não-lineares}
\label{sec:nao-linear}

Nesta seção analisaremos o caso em que a função $f_i(\vect{p})$
(equação \ref{eq:fi}) não é uma combinação linear dos parâmetros.
Neste caso, a derivada de $f_i(\vect{p})$ em relação aos parâmetros também será
uma função dos parâmetros.
Logo, dependendo das características da função $f_i(\vect{p})$, o gradiente da
função do ajuste $\phi(\vect{p})$ pode ser nulo para mais de um valor de
$\vect{p}$.
Em outras palavras, a função $\phi(\vect{p})$ pode possuir
vários pontos extremos além daquele em que esta função seja mínima.
Por esta razão, o cálculo do vetor $\opt{p}$ que minimiza a função do ajuste deve
ser feito de forma iterativa.
Isto difere do problema inverso linear e caracteriza um {\it problema inverso
não-linear}.
\\
\indent A busca pelo vetor $\opt{p}$ que minimiza uma determinada função faz
parte de uma área da matemática conhecida como {\it otimização}, dentro da qual
existem diversos métodos (KELLEY, 1999).
O procedimento padrão para realizar esta busca de forma iterativa é começar com
uma determinada aproximação inicial $\vect{p}_0$ e calcular uma correção
$\Delta\vect{p}$.
Esta correção é então aplicada à aproximação inicial dando origem a um novo vetor
$\vect{p}_1$.
Este novo vetor serve como aproximação inicial para o cálculo de um segundo
vetor $\vect{p}_2$, e assim sucessivamente.
O processo termina quando é encontrado um vetor $\est{p}$ que seja próximo ao
vetor $\opt{p}$ que minimiza a função em questão.
Em geral, não se tem garantia de que o vetor $\est{p}$ seja igual ao vetor
$\opt{p}$.
\\
\indent Dentre os vários métodos existentes, apresentaremos a seguir aquele
conhecido como método de Gauss-Newton.
O procedimento para calcular a correção $\Delta\vect{p}$ começa com a
expansão em série de Taylor da função a ser minimizada, neste caso $\phi(\vect{p})$.
A expansão é feita até segunda ordem e em torno da aproximação inicial $\vect{p}_0$
(ver Apêndice \ref{chap:opmat})

\begin{equation}
\vect{p} =  \vect{p}_0 + \Delta\vect{p} \thinspace ,
\end{equation}

\begin{equation}
\phi(\vect{p}_0 + \Delta\vect{p}) \approx \phi(\vect{p}_0) +
    \vect{\nabla}\phi(\vect{p}_0)^T\Delta\vect{p} +
    \frac{1}{2}\Delta\vect{p}^T\mat{\nabla}\phi(\vect{p}_0)\Delta\vect{p}
= \psi(\vect{p}) \thinspace ,
\label{eq:taylor}
\end{equation}

\noindent sendo $\vect{\nabla}\phi(\vect{p}_0)$ o vetor gradiente  e
$\mat{\nabla}\phi(\vect{p}_0)$ a {\it matriz Hessiana} da função $\phi(\vect{p})$,
calculados em $\vect{p}_0$.
\\
\indent A função $\psi(\vect{p})$ (equação \ref{eq:taylor}) é uma aproximação
de segunda ordem para a função $\phi(\vect{p})$ em torno do ponto $\vect{p}_0$.
Tal como no {\it problema inverso linear}, desejamos encontrar um ponto $\est{p}$
que minimiza a função $\psi(\vect{p})$, ou seja, onde seu gradiente é nulo.
O gradiente de $\psi(\vect{p})$ é dado por (ver Apêndice \ref{chap:opmat})

\begin{equation}
\vect{\nabla}\psi(\vect{p}) = \vect{\nabla}\left[
    \phi(\vect{p}_0) +
    \vect{\nabla}\phi(\vect{p}_0)^T\Delta\vect{p} +
    \frac{1}{2}\Delta\vect{p}^T\mat{\nabla}\phi(\vect{p}_0)\Delta\vect{p}
    \right] =
    \vect{\nabla}\phi(\vect{p}_0) + \mat{\nabla}\phi(\vect{p}_0)\Delta\vect{p}
    \thinspace .
\end{equation}

\noindent Logo, o incremento $\Delta\vect{p} = \est{p} - \vect{p}_0$, que leva
da aproximação inicial ao ponto $\est{p}$ onde o gradiente de $\psi(\vect{p})$ é
nulo, é a solução do sistema de equações

\begin{equation}
     \mat{\nabla}\phi(\vect{p}_0)\Delta\vect{p} = -\vect{\nabla}\phi(\vect{p}_0)
    \thinspace .
\label{eq:sistema_normal_nlin}
\end{equation}

\noindent Este sistema de equações é o {\it sistema de equações normais} do
{\it problema inverso não-linear}.
\\
\indent O cálculo do vetor gradiente de $\phi(\vect{p})$ foi feito anteriormente
(equação \ref{eq:gradphi}), restando então o cálculo da matriz Hessiana
$\mat{\nabla}\phi(\vect{p})$.
Para tanto, deriva-se o $i$-ésimo elemento do vetor gradiente
(equações \ref{eq:gradphi_partial} e \ref{eq:del_phi_del_pi_simple}) em relação
ao $j$-ésimo elemento do vetor de parâmetros $p_j$ (equação \ref{eq:param_vect})

\begin{equation}
\begin{split}
\dfrac{\partial}{\partial p_j}\left(\dfrac{\partial \phi(\vect{p})}{\partial p_i}\right)
&=
\dfrac{\partial}{\partial p_j}\left(
    -2\dfrac{\partial\vect{f}(\vect{p})}{\partial p_i}^T
    \left[\vect{d}^{\thinspace o} - \vect{f}(\vect{p})\right]
    \right)
\\[0.5cm]
&=
\underbrace{
\left( -2\dfrac{\partial^2 \vect{f}(\vect{p})}{\partial p_j \partial p_i}^T
\left[\vect{d}^{\thinspace o} - \vect{f}(\vect{p})\right]
\right)}_{\text{escalar}} +
\underbrace{
\left( 2 \dfrac{\partial\vect{f}(\vect{p})}{\partial p_i}^T
    \dfrac{\partial\vect{f}(\vect{p})}{\partial p_j} \right)}_{\text{escalar}} .
\end{split}
\label{eq:hessian_ij_exact}
\end{equation}

\indent Esta equação \ref{eq:hessian_ij_exact} representa o $j$-ésimo elemento
da $i$-ésima linha da matriz Hessiana (ver Apêndice \ref{chap:opmat}).
No método de Gauss-Newton, o termo envolvendo as segundas derivadas de
$\vect{f}(\vect{p})$ é negligenciado.
Desta forma a equação \ref{eq:hessian_ij_exact} pode ser aproximada por

\begin{equation}
\dfrac{\partial}{\partial p_j}\left(\dfrac{\partial \phi(\vect{p})}{\partial p_i}\right)
\approx 2 \dfrac{\partial\vect{f}(\vect{p})}{\partial p_i}^T
    \dfrac{\partial\vect{f}(\vect{p})}{\partial p_j} \thinspace .
\label{eq:hessian_ij_approx}
\end{equation}

\noindent Sendo assim, matriz Hessiana avaliada em $\vect{p}_0$ é dada por

\begin{equation*}
\mat{\nabla}\phi(\vect{p}_0) \approx
    2
    \begin{bmatrix}
    \dfrac{\partial\vect{f}(\vect{p}_0)}{\partial p_1}^T\dfrac{\partial\vect{f}(\vect{p}_0)}{\partial p_1} &
    \dfrac{\partial\vect{f}(\vect{p}_0)}{\partial p_1}^T\dfrac{\partial\vect{f}(\vect{p}_0)}{\partial p_2} &
    \ldots &
    \dfrac{\partial\vect{f}(\vect{p}_0)}{\partial p_1}^T\dfrac{\partial\vect{f}(\vect{p}_0)}{\partial p_M}
    \vspace{0.3cm}\\
    \dfrac{\partial\vect{f}(\vect{p}_0)}{\partial p_2}^T\dfrac{\partial\vect{f}(\vect{p}_0)}{\partial p_1} &
    \dfrac{\partial\vect{f}(\vect{p}_0)}{\partial p_2}^T\dfrac{\partial\vect{f}(\vect{p}_0)}{\partial p_2} &
    \ldots &
    \dfrac{\partial\vect{f}(\vect{p}_0)}{\partial p_2}^T\dfrac{\partial\vect{f}(\vect{p}_0)}{\partial p_M}
    \\
    \vdots & \vdots & \ddots & \vdots
    \\
    \dfrac{\partial\vect{f}(\vect{p}_0}{\partial p_M}^T\dfrac{\partial\vect{f}(\vect{p}_0)}{\partial p_1} &
    \dfrac{\partial\vect{f}(\vect{p}_0)}{\partial p_M}^T\dfrac{\partial\vect{f}(\vect{p}_0)}{\partial p_2} &
    \ldots &
    \dfrac{\partial\vect{f}(\vect{p}_0)}{\partial p_M}^T\dfrac{\partial\vect{f}(\vect{p}_0)}{\partial p_M}
    \end{bmatrix} ,
\end{equation*}

\noindent ou

\begin{equation}
\mat{\nabla}\phi(\vect{p}_0) \approx
    2
    \begin{bmatrix}
    \dfrac{\partial\vect{f}(\vect{p}_0)}{\partial p_1}^T \vspace{0.3cm}\\
    \dfrac{\partial\vect{f}(\vect{p}_0)}{\partial p_2}^T \\
    \vdots \\
    \dfrac{\partial\vect{f}(\vect{p}_0)}{\partial p_M}^T
    \end{bmatrix}
    \begin{bmatrix}
    \dfrac{\partial\vect{f}(\vect{p}_0)}{\partial p_1} &
    \dfrac{\partial\vect{f}(\vect{p}_0)}{\partial p_2} &
    \ldots &
    \dfrac{\partial\vect{f}(\vect{p}_0)}{\partial p_M}
    \end{bmatrix}
    =
    \mat{G}(\vect{p}_0)^T\mat{G}(\vect{p}_0) \thinspace ,
\label{eq:hessian_approx}
\end{equation}

\noindent em que $\mat{G}(\vect{p}_0)$ é a {\it matriz de sensibilidade}
(equação \ref{eq:jacobian}) avaliada em $\vect{p}_0$.
\\
\indent A partir das equações \ref{eq:gradphi} e \ref{eq:hessian_approx}, o
sistema de equações \ref{eq:sistema_normal_nlin} pode ser escrito como

\begin{equation}
    \mat{G}(\vect{p}_0)^T\mat{G}(\vect{p}_0)\Delta\vect{p} =
        \mat{G}(\vect{p}_0)^T \left[
            \vect{d}^{\thinspace o} - \vect{f}(\vect{p}_0) \right]
    \thinspace ,
\label{eq:sistema_normal_gaussnewton}
\end{equation}

\noindent em que $\vect{d}^{\thinspace o}$ é o vetor de dados observados e
$\vect{f}(\vect{p}_0)$ é o vetor de dados preditos avaliado em $\vect{p}_0$.
\\
\indent A equação \ref{eq:sistema_normal_gaussnewton} descreve o cálculo da
correção $\Delta\vect{p}$ em uma determinada iteração do método Gauss-Newton.
Para o caso em que a função $f_i(\vect{p})$ é uma combinação linear dos
parâmetros (equação \ref{eq:comb_linear}), a equação
\ref{eq:sistema_normal_gaussnewton} se reduz ao sistema
normal do {\it problema inverso linear} (equação \ref{eq:sistema_normal})
\footnote{Dica: Lembre que nos casos lineares $\vect{f}(\vect{p})=\mat{G}\vect{p}+\vect{b}$.}.
Isto mostra que a solução do problema inverso linear é um {\it caso particular}
da solução do {\it problema inverso não-linear} pelo método Gauss-Newton.
Outra observação importante é a semelhança entre as equações
\ref{eq:sistema_normal} e \ref{eq:sistema_normal_gaussnewton}, o que evidencia
que um problema inverso não-linear é uma sucessão de problemas inversos lineares.

