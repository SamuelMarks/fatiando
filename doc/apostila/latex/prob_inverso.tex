\chapter{Formulação matemática do problema inverso}

Dado um conjunto de $N$ observações, feitas em diferentes posições, tempos, etc.,
de\-fi\-ni\-mos um {\it vetor de dados observados}

\begin{equation}
\vect{d}^{\thinspace o} =
    \begin{bmatrix}
    d_1^{\thinspace o} \\
    d_2^{\thinspace o} \\
    \vdots \\
    d_N^{\thinspace o}
    \end{bmatrix},
\end{equation}

\noindent em que $d_i^{\thinspace o}$, $i = 1, 2, 3, \dotsc, N$, é o dado
observado na $i$-ésima posição, tempo, etc.
De forma análoga, definimos um {\it vetor de dados preditos}

\begin{equation}
\vect{d}^{\thinspace p} =
    \begin{bmatrix}
    d_1^{\thinspace p} \\
    d_2^{\thinspace p} \\
    \vdots \\
    d_N^{\thinspace p}
    \end{bmatrix},
\end{equation}

\noindent em que $d_i^{\thinspace p}$, $i = 1, 2, 3, \dotsc, N$, é o dado predito
calculado na mesma posição, tempo, etc., que o $i$-ésimo dado observado.
Continuando no espírito de definição de vetores, definimos também um vetor que
agrupa todos os $M$ parâmetros, denominado {\it vetor de parâmetros}

\begin{equation}
\vect{p} =
    \begin{bmatrix}
    p_1 \\
    p_2 \\
    \vdots \\
    p_M
    \end{bmatrix},
\label{eq:param_vect}
\end{equation}

\noindent em que $p_j$, $j = 1, 2, 3, \dotsc, M$, é o $j$-ésimo parâmetro.
\\
\indent Como vimos no Capítulo \ref{chap:intro}, os dados preditos são descritos
por uma função dos parâmetros, ou seja,

\begin{equation}
d^{\thinspace p}_i = f_i(\vect{p})\thinspace.
\label{eq:fi}
\end{equation}

\noindent Desta forma, podemos dizer que o vetor de dados preditos é uma função
dos parâmetros

\begin{equation}
\vect{d}^{\thinspace p}= \vect{f}(\vect{p}) =
    \begin{bmatrix}
    f_1(\vect{p}) \\
    f_2(\vect{p}) \\
    \vdots \\
    f_N(\vect{p})
    \end{bmatrix}.
\label{eq:dados_preditos}
\end{equation}

\indent O problema inverso consiste em encontrar um vetor de parâmetros $\vect{p}$
que produza os dados preditos mais próximos possívies dos dados observados.
Para determinar a ``proximidade'' entre os dados observados e os dados preditos,
é necessário quantificar a distância entre eles.
Isto é feito em termos da norma do {\it vetor de resíduos}

\begin{equation}
\vect{r} = \vect{d}^{\thinspace o} - \vect{f}(\vect{p}),
\end{equation}

\noindent em que $\vect{d}^{\thinspace o}$ é o vetor de {\it dados observados}
e $\vect{f}(\vect{p})$ é o vetor de {\it dados preditos}.
\\
\indent Para quantificar a distância entre os dados observados e os dados
preditos utiliza-se, usualmente, o quadrado da
norma quadrática (também conhecida como norma $\ell_2$ ou norma Euclidiana)
do vetor de resíduos

\begin{equation}
\norm{\vect{r}}_2^2 =
    \sum\limits_{i=1}^N \left[d^{\thinspace o}_i - f_i(\vect{p})\right]^2 \, .
\label{eq:norma_l2}
\end{equation}

\noindent Esta equação é também uma função escalar
dos parâmetros. Assim sendo, definimos uma função $\phi(\vect{p})$, chamada de
{\it função do ajuste}, como

\begin{equation}
\phi(\vect{p}) = \norm{\vect{r}}_2^2 =
    \sum\limits_{i=1}^N \left[d^{\thinspace o}_i - f_i(\vect{p})\right]^2 \, .
\label{eq:ajuste_sum}
\end{equation}

\indent Lembramos que o quadrado da norma Euclidiana de um vetor é igual ao
produto escalar do vetor com ele mesmo, ou seja,

\begin{equation}
\norm{\vect{r}}_2^2 = \vect{r} \cdot \vect{r} =
    r_1r_1 + r_2r_2 + \dotsb + r_Nr_N \thinspace .
\label{eq:dotprod}
\end{equation}

\noindent Como o vetor $\vect{r}$ é um {\it vetor coluna} (matriz com uma
coluna), podemos escrever o produto escalar da equação \ref{eq:dotprod} como

\begin{equation}
\vect{r} \cdot \vect{r} = \vect{r}^T \vect{r} =
    \begin{bmatrix}
        r_1 & r_2 & \ldots & r_N
    \end{bmatrix}
    \begin{bmatrix}
        r_1 \\ r_2 \\ \vdots \\ r_N
    \end{bmatrix} .
\label{eq:vectdotprod}
\end{equation}

\noindent Dessa forma podemos reescrever a função do ajuste (equação
\ref{eq:ajuste_sum}) como

\begin{equation}
\phi(\vect{p}) = \vect{r}^T \vect{r} =
    \left[\vect{d}^{\thinspace o} - \vect{f}(\vect{p})\right]^T
    \left[\vect{d}^{\thinspace o} - \vect{f}(\vect{p})\right] .
\label{eq:ajuste}
\end{equation}

\indent Neste ponto é importante ressaltar o seguinte conceito:

\begin{quote}
{\tt A {\bf função do ajuste} é uma função escalar que
quantifica a dis\-tân\-cia entre os dados observados e os dados preditos para um
de\-ter\-mi\-na\-do vetor de parâmetros $\vect{p}$.}
\end{quote}

\indent Perante este conceito, o problema inverso consiste em determinar um
vetor $\opt{p}$ que minimiza a função do ajuste $\phi(\vect{p})$. 
Matematicamente, isso equivale a encontrar o vetor $\opt{p}$ tal que o gradiente
da função $\phi(\vect{p})$ avaliado em $\opt{p}$ seja igual ao vetor nulo.
Este vetor $\opt{p}$ é um {\it ponto extremo} da função
$\phi(\vect{p})$.
\\
\indent O gradiente da função $\phi(\vect{p})$, avaliado em um $\vect{p}$ qualquer,
é um vetor $M$-dimensional definido como (ver Apêndice \ref{chap:opmat})

\begin{equation}
\vect{\nabla} \phi(\vect{p}) =
    \begin{bmatrix}
        \dfrac{\partial \phi(\vect{p})}{\partial p_1} \vspace{0.3cm}\\
        \dfrac{\partial \phi(\vect{p})}{\partial p_2}\\
        \vdots \\
        \dfrac{\partial \phi(\vect{p})}{\partial p_M}
    \end{bmatrix} ,
\label{eq:gradphi_partial}
\end{equation} 

\noindent em que $\vect{\nabla}$ é o operador gradiente (Apêndice \ref{chap:opmat}).
A partir da equação \ref{eq:ajuste}, a expressão para o $i$-ésimo elemento do
vetor gradiente é

\begin{equation}
\begin{split}
\dfrac{\partial \phi(\vect{p})}{\partial p_i} &=
    \dfrac{\partial}{\partial p_i}\left\{
        \left[\vect{d}^{\thinspace o} - \vect{f}(\vect{p})\right]^T
        \left[\vect{d}^{\thinspace o} - \vect{f}(\vect{p})\right]
    \right\} \\[0.5cm]
    &= 
    \underbrace{
    \left\{-\dfrac{\partial\vect{f}(\vect{p})}{\partial p_i}^T
            \left[\vect{d}^{\thinspace o} - \vect{f}(\vect{p})\right]
    \right\}}_{\text{escalar}}
    +
    \underbrace{
    \left\{-\left[\vect{d}^{\thinspace o} - \vect{f}(\vect{p})\right]^T            
            \dfrac{\partial\vect{f}(\vect{p})}{\partial p_i}
    \right\}}_{\text{escalar}}
\end{split} .
\label{eq:del_phi_del_pi}
\end{equation}

\indent Lembrando que o transposto de um escalar é igual a ele mesmo, podemos
tomar o transposto do segundo termo do lado direito da equação
\ref{eq:del_phi_del_pi}, obtendo

\begin{equation}
\dfrac{\partial \phi(\vect{p})}{\partial p_i} = 
    -2\dfrac{\partial\vect{f}(\vect{p})}{\partial p_i}^T
    \left[\vect{d}^{\thinspace o} - \vect{f}(\vect{p})\right],
\label{eq:del_phi_del_pi_simple}
\end{equation}

\noindent em que $\dfrac{\partial\vect{f}(\vect{p})}{\partial p_i}$ é um vetor
$N$-dimensional (ver Apêndice \ref{chap:opmat}) dado por

\begin{equation}
\dfrac{\partial\vect{f}(\vect{p})}{\partial p_i} =
    \begin{bmatrix}
        \dfrac{\partial f_1(\vect{p})}{\partial p_i} \vspace{0.3cm}\\
        \dfrac{\partial f_2(\vect{p})}{\partial p_i}\\
        \vdots \\
        \dfrac{\partial f_N(\vect{p})}{\partial p_i}
    \end{bmatrix}.
\label{eq:del_f_del_pi}
\end{equation}

\indent Substituindo as equações \ref{eq:del_phi_del_pi_simple} e
\ref{eq:del_f_del_pi} na expressão do gradiente (equação \ref{eq:gradphi_partial})
obtemos

\begin{equation}
\begin{split}
\vect{\nabla} \phi(\vect{p}) &=
        \begin{bmatrix}
            -2\dfrac{\partial\vect{f}(\vect{p})}{\partial p_1}^T
                \left[\vect{d}^{\thinspace o} - \vect{f}(\vect{p})\right]
                \vspace{0.3cm} \\
            -2\dfrac{\partial\vect{f}(\vect{p})}{\partial p_2}^T
                \left[\vect{d}^{\thinspace o} - \vect{f}(\vect{p})\right]\\
            \vdots \\
            -2\dfrac{\partial\vect{f}(\vect{p})}{\partial p_M}^T
                \left[\vect{d}^{\thinspace o} - \vect{f}(\vect{p})\right]
        \end{bmatrix}
    \\[0.5cm] &=
        -2
        \begin{bmatrix}
            \dfrac{\partial\vect{f}(\vect{p})}{\partial p_1}^T\vspace{0.3cm} \\
            \dfrac{\partial\vect{f}(\vect{p})}{\partial p_2}^T \\
            \vdots \\
            \dfrac{\partial\vect{f}(\vect{p})}{\partial p_M}^T
        \end{bmatrix}
        \left[\vect{d}^{\thinspace o} - \vect{f}(\vect{p})\right].
\end{split}
\label{eq:gradphi_partial_f}
\end{equation}

\noindent Por fim, o gradiente da função do ajuste $\phi(\vect{p})$ pode ser
escrito como

\begin{equation}
\vect{\nabla} \phi(\vect{p}) = -2\mat{G}(\vect{p})^T
    \left[\vect{d}^{\thinspace o} - \vect{f}(\vect{p})\right].
\label{eq:gradphi}
\end{equation}

\noindent em que 

\begin{equation}
\mat{G}(\vect{p}) = 
\begin{bmatrix}
    \dfrac{\partial\vect{f}(\vect{p})}{\partial p_1} &
    \dfrac{\partial\vect{f}(\vect{p})}{\partial p_2} &
    \ldots &
    \dfrac{\partial\vect{f}(\vect{p})}{\partial p_M}
\end{bmatrix}
=
\begin{bmatrix}
    \dfrac{\partial f_1(\vect{p})}{\partial p_1} &
        \dfrac{\partial f_1(\vect{p})}{\partial p_2} &
        \ldots &
        \dfrac{\partial f_1(\vect{p})}{\partial p_M}
    \vspace{0.3cm}\\
    \dfrac{\partial f_2(\vect{p})}{\partial p_1} &
        \dfrac{\partial f_2(\vect{p})}{\partial p_2} &
        \ldots & 
        \dfrac{\partial f_2(\vect{p})}{\partial p_M}
    \\
    \vdots & \vdots & \ddots & \vdots
    \\
    \dfrac{\partial f_N(\vect{p})}{\partial p_1} &
        \dfrac{\partial f_N(\vect{p})}{\partial p_2} &
        \ldots & 
        \dfrac{\partial f_N(\vect{p})}{\partial p_M}        
\end{bmatrix}.
\label{eq:jacobian}
\end{equation}

\noindent A matriz $\mat{G}(\vect{p})$ de dimensão $N \times M$ é a
{\it matriz Jacobiana} de $\vect{f}(\vect{p})$.

\begin{quote}
{\tt Em problemas inversos, essa matriz $\mat{G}$ é comumente denominada
{\bf matriz de sensibilidade}, uma vez que o $i$-ésimo elemento de sua $j$-ésima
coluna expressa a sensibilidade do $i$-ésimo dado pre\-di\-to em re\-la\-ção à variações
do $j$-ésimo parâmetro.}
\end{quote}

\indent As equações \ref{eq:gradphi} e \ref{eq:jacobian} mostram que o gradiente
da função do ajuste $\phi(\vect{p})$ depende do vetor de dados preditos
$\vect{f}(\vect{p})$ (equação \ref{eq:dados_preditos}) e sua derivada em relação
aos parâmetros $\vect{p}$.
Sendo assim, a função $f$ que relaciona os parâmetros aos dados preditos determina
o comportamento do gradiente de $\phi(\vect{p})$.
Nas próximas seções analisaremos os casos em que a função $f$ é linear ou
não-linear em relação aos parâmetros.
Essa análise nos permitirá compreender a influência da função $f$ na busca pelo
vetor de parâmetros $\opt{p}$ que minimiza a função $\phi(\vect{p})$.

\section{Problemas lineares}

Se a função $f_i(\vect{p})$ (equação \ref{eq:fi}), que relaciona o vetor de
parâmetros $\vect{p}$ ao $i$-ésimo dado predito, for uma combinação linear dos
parâmetros, ela terá a seguinte forma

\begin{equation}
f_i(\vect{p}) = g_{i1}p_1 + g_{i2}p_2 + \dotsb + g_{iM}p_M + b_i \thinspace ,
\label{eq:comb_linear}
\end{equation}

\noindent em que $g_{ij}$, $j = 1, 2, \cdots, M$, e $b_i$ são constantes.
Portanto, a derivada de $f_i(\vect{p})$ em relação ao $j$-ésimo parâmetro
$p_j$ é dada por

\begin{equation}
\dfrac{\partial f_i(\vect{p})}{\partial p_j} = g_{ij} \thinspace ,
\end{equation}

\noindent que não depende dos parâmetros. Para um conjunto de $N$ dados, o vetor
de dados preditos tem a seguinte forma

\begin{equation}
\vect{f}(\vect{p}) = \mat{G}\vect{p} + \vect{b} \thinspace ,
\label{eq:f_igual_Gp}
\end{equation}

\noindent sendo $\mat{G}$ a matriz de sensibilidade (equação \ref{eq:jacobian})
e $\vect{b}$ um vetor de constantes.
Substituindo a expressão acima no gradiente da função do ajuste (equação
\ref{eq:gradphi}), chegamos a

\begin{equation}
\vect{\nabla}\phi(\vect{p}) = -2\mat{G}^T \left(\vect{d}^{\thinspace o} -
    \mat{G}\vect{p} - \vect{b} \right).
\end{equation}

\indent Seja $\opt{p}$ o vetor que minimiza a função $\phi(\vect{p})$, o
gradiente desta função avaliado em $\opt{p}$ é igual ao vetor nulo.
Isto nos leva ao sistema de equações

\begin{equation}
\mat{G}^T\mat{G}\opt{p} = \mat{G}^T\left(\vect{d}^{\thinspace o} - \vect{b} \right).
\label{eq:sistema_normal}
\end{equation}

\noindent Este sistema é conhecido como {\it sistema de equações normais}. A
solução para este sistema é

\begin{equation}
\opt{p} = (\mat{G}^T\mat{G})^{-1} \mat{G}^T
    \left(\vect{d}^{\thinspace o} - \vect{b} \right),
\label{eq:estimador_mq}
\end{equation}

\noindent que é conhecido como {\it estimador de mínimos quadrados}.
\\
\indent As equações \ref{eq:sistema_normal} e \ref{eq:estimador_mq} mostram que 
o vetor $\opt{p}$ que minimiza a função $\phi(\vect{p})$ pode ser
obtido diretamente a partir da matriz $\mat{G}$ e dos vetores $\vect{b}$ e
$\vect{d}^{\thinspace o}$. Isto caracteriza um {\it problema inverso linear}.

\section{Problemas não-lineares}
\label{sec:nao-linear}

Nesta seção analisaremos o caso em que a função $f_i(\vect{p})$
(equação \ref{eq:fi}) não é uma combinação linear dos parâmetros.
Neste caso, a derivada de $f_i(\vect{p})$ em relação aos parâmetros também será
uma função dos parâmetros.
Logo, dependendo das características da função $f_i(\vect{p})$, o gradiente da
função do ajuste $\phi(\vect{p})$ pode ser nulo para mais de um valor de
$\vect{p}$.
Em outras palavras, a função $\phi(\vect{p})$ pode possuir
vários pontos extremos além daquele em que esta função seja mínima.
Por esta razão, o cálculo do vetor $\opt{p}$ que minimiza a função do ajuste deve
ser feito de forma iterativa.
Isto difere do problema inverso linear e caracteriza um {\it problema inverso
não-linear}.
\\
\indent A busca pelo vetor $\opt{p}$ que minimiza uma determinada função faz
parte de uma área da matemática conhecida como {\it otimização}, dentro da qual
existem diversos métodos (KELLEY, 1999).
O procedimento padrão para realizar esta busca de forma iterativa é começar com
uma determinada aproximação inicial $\vect{p}_0$ e calcular uma correção
$\Delta\vect{p}$.
Esta correção é então aplicada à aproximação inicial dando origem a um novo vetor
$\vect{p}_1$.
Este novo vetor serve como aproximação inicial para o cálculo de um segundo
vetor $\vect{p}_2$, e assim sucessivamente.
O processo termina quando é encontrado um vetor $\est{p}$ que seja próximo ao
vetor $\opt{p}$ que minimiza a função em questão.
Em geral, não se tem garantia de que o vetor $\est{p}$ seja igual ao vetor
$\opt{p}$.
\\
\indent Dentre os vários métodos existentes, apresentaremos a seguir aquele
conhecido como método de Gauss-Newton.
O procedimento para calcular a correção $\Delta\vect{p}$ começa com a
expansão em série de Taylor da função a ser minimizada, neste caso $\phi(\vect{p})$.
A expansão é feita até segunda ordem e em torno da aproximação inicial $\vect{p}_0$
(ver Apêndice \ref{chap:opmat})

\begin{equation}
\vect{p} =  \vect{p}_0 + \Delta\vect{p} \thinspace ,
\end{equation}

\begin{equation}
\phi(\vect{p}_0 + \Delta\vect{p}) \approx \phi(\vect{p}_0) +
    \vect{\nabla}\phi(\vect{p}_0)^T\Delta\vect{p} +
    \dfrac{1}{2}\Delta\vect{p}^T\mat{\nabla}\phi(\vect{p}_0)\Delta\vect{p}
= \psi(\vect{p}) \thinspace ,
\label{eq:taylor}
\end{equation}

\noindent sendo $\vect{\nabla}\phi(\vect{p}_0)$ o vetor gradiente  e
$\mat{\nabla}\phi(\vect{p}_0)$ a {\it matriz Hessiana} da função $\phi(\vect{p})$,
calculados em $\vect{p}_0$.
\\
\indent A função $\psi(\vect{p})$ (equação \ref{eq:taylor}) é uma aproximação
de segunda ordem para a função $\phi(\vect{p})$ em torno do ponto $\vect{p}_0$.
Tal como no {\it problema inverso linear}, desejamos encontrar um ponto $\est{p}$
que minimiza a função $\psi(\vect{p})$, ou seja, onde seu gradiente é nulo.
O gradiente de $\psi(\vect{p})$ é dado por (ver Apêndice \ref{chap:opmat})

\begin{equation}
\vect{\nabla}\psi(\vect{p}) = \vect{\nabla}\left[
    \phi(\vect{p}_0) +
    \vect{\nabla}\phi(\vect{p}_0)^T\Delta\vect{p} +
    \dfrac{1}{2}\Delta\vect{p}^T\mat{\nabla}\phi(\vect{p}_0)\Delta\vect{p}
    \right] =
    \vect{\nabla}\phi(\vect{p}_0) + \mat{\nabla}\phi(\vect{p}_0)\Delta\vect{p}
    \thinspace .
\end{equation}

\noindent Logo, o incremento $\Delta\vect{p} = \est{p} - \vect{p}_0$, que leva
da aproximação inicial ao ponto $\est{p}$ onde o gradiente de $\psi(\vect{p})$ é
nulo, é a solução do sistema de equações

\begin{equation}
     \mat{\nabla}\phi(\vect{p}_0)\Delta\vect{p} = -\vect{\nabla}\phi(\vect{p}_0)
    \thinspace .
\label{eq:sistema_normal_nlin}
\end{equation}

\noindent Este sistema de equações é o {\it sistema de equações normais} do
{\it problema inverso não-linear}.
\\
\indent O cálculo do vetor gradiente de $\phi(\vect{p})$ foi feito anteriormente
(equação \ref{eq:gradphi}), restando então o cálculo da matriz Hessiana
$\mat{\nabla}\phi(\vect{p})$.
Para tanto, deriva-se o $i$-ésimo elemento do vetor gradiente
(equações \ref{eq:gradphi_partial} e \ref{eq:del_phi_del_pi_simple}) em relação
ao $j$-ésimo elemento do vetor de parâmetros $p_j$ (equação \ref{eq:param_vect})

\begin{equation}
\begin{split}
\dfrac{\partial}{\partial p_j}\left(\dfrac{\partial \phi(\vect{p})}{\partial p_i}\right)
&=
\dfrac{\partial}{\partial p_j}\left(
    -2\dfrac{\partial\vect{f}(\vect{p})}{\partial p_i}^T
    \left[\vect{d}^{\thinspace o} - \vect{f}(\vect{p})\right]
    \right)
\\[0.5cm]
&=
\underbrace{
\left( -2\dfrac{\partial^2 \vect{f}(\vect{p})}{\partial p_j \partial p_i}^T
\left[\vect{d}^{\thinspace o} - \vect{f}(\vect{p})\right]
\right)}_{\text{escalar}} +
\underbrace{
\left( 2 \dfrac{\partial\vect{f}(\vect{p})}{\partial p_i}^T
    \dfrac{\partial\vect{f}(\vect{p})}{\partial p_j} \right)}_{\text{escalar}} .
\end{split}
\label{eq:hessian_ij_exact}
\end{equation}

\indent Esta equação \ref{eq:hessian_ij_exact} representa o $j$-ésimo elemento
da $i$-ésima linha da matriz Hessiana (ver Apêndice \ref{chap:opmat}).
No método de Gauss-Newton, o termo envolvendo as segundas derivadas de
$\vect{f}(\vect{p})$ é negligenciado.
Desta forma a equação \ref{eq:hessian_ij_exact} pode ser aproximada por

\begin{equation}
\dfrac{\partial}{\partial p_j}\left(\dfrac{\partial \phi(\vect{p})}{\partial p_i}\right)
\approx 2 \dfrac{\partial\vect{f}(\vect{p})}{\partial p_i}^T
    \dfrac{\partial\vect{f}(\vect{p})}{\partial p_j} \thinspace .
\label{eq:hessian_ij_approx}
\end{equation}

\noindent Sendo assim, matriz Hessiana avaliada em $\vect{p}_0$ é dada por

\begin{equation*}
\mat{\nabla}\phi(\vect{p}_0) \approx
    2
    \begin{bmatrix}
    \dfrac{\partial\vect{f}(\vect{p}_0)}{\partial p_1}^T\dfrac{\partial\vect{f}(\vect{p}_0)}{\partial p_1} &
    \dfrac{\partial\vect{f}(\vect{p}_0)}{\partial p_1}^T\dfrac{\partial\vect{f}(\vect{p}_0)}{\partial p_2} &
    \ldots &
    \dfrac{\partial\vect{f}(\vect{p}_0)}{\partial p_1}^T\dfrac{\partial\vect{f}(\vect{p}_0)}{\partial p_M}
    \vspace{0.3cm}\\
    \dfrac{\partial\vect{f}(\vect{p}_0)}{\partial p_2}^T\dfrac{\partial\vect{f}(\vect{p}_0)}{\partial p_1} &
    \dfrac{\partial\vect{f}(\vect{p}_0)}{\partial p_2}^T\dfrac{\partial\vect{f}(\vect{p}_0)}{\partial p_2} &
    \ldots &
    \dfrac{\partial\vect{f}(\vect{p}_0)}{\partial p_2}^T\dfrac{\partial\vect{f}(\vect{p}_0)}{\partial p_M}
    \\
    \vdots & \vdots & \ddots & \vdots
    \\
    \dfrac{\partial\vect{f}(\vect{p}_0}{\partial p_M}^T\dfrac{\partial\vect{f}(\vect{p}_0)}{\partial p_1} &
    \dfrac{\partial\vect{f}(\vect{p}_0)}{\partial p_M}^T\dfrac{\partial\vect{f}(\vect{p}_0)}{\partial p_2} &
    \ldots &
    \dfrac{\partial\vect{f}(\vect{p}_0)}{\partial p_M}^T\dfrac{\partial\vect{f}(\vect{p}_0)}{\partial p_M}
    \end{bmatrix} ,
\end{equation*}

\noindent ou

\begin{equation}
\mat{\nabla}\phi(\vect{p}_0) \approx
    2
    \begin{bmatrix}
    \dfrac{\partial\vect{f}(\vect{p}_0)}{\partial p_1}^T \vspace{0.3cm}\\
    \dfrac{\partial\vect{f}(\vect{p}_0)}{\partial p_2}^T \\
    \vdots \\
    \dfrac{\partial\vect{f}(\vect{p}_0)}{\partial p_M}^T
    \end{bmatrix}
    \begin{bmatrix}
    \dfrac{\partial\vect{f}(\vect{p}_0)}{\partial p_1} &
    \dfrac{\partial\vect{f}(\vect{p}_0)}{\partial p_2} &
    \ldots &
    \dfrac{\partial\vect{f}(\vect{p}_0)}{\partial p_M}
    \end{bmatrix}
    =
    \mat{G}(\vect{p}_0)^T\mat{G}(\vect{p}_0) \thinspace ,
\label{eq:hessian_approx}
\end{equation}

\noindent em que $\mat{G}(\vect{p}_0)$ é a {\it matriz de sensibilidade}
(equação \ref{eq:jacobian}) avaliada em $\vect{p}_0$.
\\
\indent A partir das equações \ref{eq:gradphi} e \ref{eq:hessian_approx}, o
sistema de equações \ref{eq:sistema_normal_nlin} pode ser escrito como

\begin{equation}
    \mat{G}(\vect{p}_0)^T\mat{G}(\vect{p}_0)\Delta\vect{p} =
        -\mat{G}(\vect{p}_0)^T \left[
            \vect{d}^{\thinspace o} - \vect{f}(\vect{p}_0) \right]
    \thinspace ,
\label{eq:sistema_normal_gaussnewton}
\end{equation}

\noindent em que $\vect{d}^{\thinspace o}$ é o vetor de dados observados e
$\vect{f}(\vect{p}_0)$ é o vetor de dados preditos avaliado em $\vect{p}_0$.
\\
\indent A equação \ref{eq:sistema_normal_gaussnewton} descreve o cálculo da
correção $\Delta\vect{p}$ em uma determinada iteração do método Gauss-Newton.
Para o caso em que a função $f_i(\vect{p})$ é uma combinação linear dos
parâmetros (equação \ref{eq:comb_linear}), a equação
\ref{eq:sistema_normal_gaussnewton} se reduz ao sistema
normal do {\it problema inverso linear} (equação \ref{eq:sistema_normal}).
Isto mostra que a solução do problema inverso linear é um {\it caso particular}
da solução do {\it problema inverso não-linear} pelo método Gauss-Newton.
Outra observação importante é a semelhança entre as equações
\ref{eq:sistema_normal} e \ref{eq:sistema_normal_gaussnewton}, o que evidencia
que um problema inverso não-linear é uma sucessão de problemas inversos lineares.


